\begin{problem}{Count $a\times b$}{standard input}{standard output}{1s}{light blue}

Marry likes to count the number of ways to choose two non-negative integers $a$ and $b$ less than $m$ to make $a\times b$ mod $m \neq 0$.

Let's denote $f(m)$ as the number of ways to choose two non-negative integers $a$ and $b$ less than $m$ to make $a\times b$ mod $m \neq 0$.

She has calculated a lot of $f(m)$ for different $m$, and now she is interested in another function $g(n)=\sum\limits_{m|n}f(m)$. For example, $g(6)=f(1)+f(2)+f(3)+f(6)=0+1+4+21=26$. She needs you to double check the answer.

\begin{minipage}{0.18\linewidth}
	\centering
	\begin{tabular}{|c|c|}\hline
	\backslashbox{a\kern-0.75em}{\kern-0.75em b}&0\\\hline
	0&0\\\hline
	\end{tabular}
	\captionof{table}{$a\times b$ mod 1}
\end{minipage}
\hfill
\begin{minipage}{0.18\linewidth}
	\centering
	\begin{tabular}{|c|c|c|}\hline
	\backslashbox{a\kern-0.75em}{\kern-0.75em b}&0&1\\\hline
	0&0&0\\\hline
	1&0&1\\\hline
	\end{tabular}
	\captionof{table}{$a\times b$ mod 2}
\end{minipage}
\hfill
\begin{minipage}{0.18\linewidth}
	\centering
	\begin{tabular}{|c|c|c|c|}\hline
	\backslashbox{a\kern-0.75em}{\kern-0.75em b}&0&1&2\\\hline
	0&0&0&0\\\hline
	1&0&1&2\\\hline
	2&0&2&1\\\hline
	\end{tabular}
	\captionof{table}{$a\times b$ mod 3}
\end{minipage}
\hfill
\begin{minipage}{0.36\linewidth}
	\centering
	\begin{tabular}{|c|c|c|c|c|c|c|}\hline
	\backslashbox{a\kern-0.75em}{\kern-0.75em b}&0&1&2&3&4&5\\\hline
	0&0&0&0&0&0&0\\\hline
	1&0&1&2&3&4&5\\\hline
	2&0&2&4&0&2&4\\\hline
	3&0&3&0&3&0&3\\\hline
	4&0&4&2&0&4&2\\\hline
	5&0&5&4&3&2&1\\\hline
	\end{tabular}
	\captionof{table}{$a\times b$ mod 6}
\end{minipage}

Give you $n$. Your task is to find $g(n)$ modulo $2^{64}$. 

\InputFile
The first line contains an integer $T$ indicating the total number of test cases.
Each test case is a line with a positive integer $n$.

\begin{itemize}
\item $1 \le T \le 20000$
\item $1 \le n \le 10^9$
\end{itemize}

\OutputFile
For each test case, print one integer $s$, representing $g(n)$ modulo $2^{64}$. 

\Example

\begin{example}
\exmp{
2
6
514
}{
26
328194
}%
\end{example}

\end{problem}