\begin{problem}{Partial Tree}{standard input}{standard output}{1s}{rose red}

In mathematics, and more specifically in graph theory, a tree is an undirected graph in which any two nodes are connected by exactly one path. In other words, any connected graph without simple cycles is a tree.

You find a partial tree on the way home. This tree has $n$ nodes but lacks of $n-1$ edges.
You want to complete this tree by adding $n-1$ edges.
There must be exactly one path between any two nodes after adding.
As you know, there are $n^{n-2}$ ways to complete this tree,
and you want to make the completed tree as cool as possible.
The coolness of a tree is the sum of coolness of its nodes.
The coolness of a node is $f(d)$, where $f$ is a predefined function and $d$ is the degree of this node.
What's the maximum coolness of the completed tree?

\InputFile
The first line contains an integer $T$ indicating the total number of test cases.
Each test case starts with an integer $n$ in one line,
then one line with $n - 1$ integers $f(1), f(2), \ldots, f(n-1)$.
\begin{itemize}
\item $1 \le T \le 2015$
\item $2 \le n \le 2015$
\item $0 \le f(i) \le 10000$
\item There are at most $10$ test cases with $n > 100$.
\end{itemize}

\OutputFile
For each test case, please output the maximum coolness of the completed tree in one line.

\Example

\begin{example}
\exmp{
2
3
2 1
4
5 1 4
}{
5
19
}%
\end{example}
\end{problem}