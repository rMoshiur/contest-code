\begin{problem}{Chip Factory}{standard input}{standard output}{9s}{dark blue}

John is a manager of a CPU chip factory, the factory produces lots of chips everyday. To manage large amounts of products, every processor has a serial number. More specifically, the factory produces $n$ chips today, the $i$-th chip produced this day has a serial number $s_i$.

At the end of the day, he packages all the chips produced this day, and send it to wholesalers. More specially, he writes a checksum number on the package, this checksum is defined as below:

$$\max_{i,j,k} (s_i+s_j) \oplus s_k$$

which $i,j,k$ are three {\bf different} integers between $1$ and $n$. And $\oplus$ is symbol of bitwise XOR.

Can you help John calculate the checksum number of today?

\InputFile
The first line of input contains an integer $T$ indicating the total number of test cases.

The first line of each test case is an integer $n$, indicating the number of chips produced today. The next line has $n$ integers $s_1, s_2, .., s_n$, separated with single space, indicating serial number of each chip.
\begin{itemize}
\item $1 \le T \le 1000$
\item $3 \le n \le 1000$
\item $0 \le s_i \le 10^9$
\item There are at most $10$ testcases with $n > 100$ 
\end{itemize}

\OutputFile
For each test case, please output an integer indicating the checksum number in a line.

\Example

\begin{example}
\exmp{
2
3
1 2 3
3
100 200 300
}{
6
400
}%
\end{example}
\end{problem}